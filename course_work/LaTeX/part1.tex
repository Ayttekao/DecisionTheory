\documentclass[a4paper,14pt]{extarticle}
\usepackage[T2A]{fontenc}
\usepackage[utf8]{inputenc}
\usepackage[russian]{babel}

\usepackage[dvips]{graphicx}
\usepackage{color}
\usepackage[dvips]{hyperref}

\usepackage{setspace}
\usepackage{indentfirst}
\usepackage{textcomp}
\usepackage{ifthen}
\usepackage{calc}

\usepackage{amssymb}

\usepackage[cache=false]{minted}


\begin{document}
\newpage

\part{Теоретическая}
\section{Задание \textnumero 1}

Предположим, что вам необходимо предсказать последовательность $Y_1, Y_2, \ldots \in {0, 1}$ независимых и одинаково распределенных случайных величин с неизвестным распределением. Пусть $D = [0, 1]$, $l(x, y) = |x - y|$. Придумайте свой алгоритм для решения этой задачи. Оцените средний ожидаемый регрет своего алгоритма по сравнению с двумя экспертами, один из которых всегда выбирает исход $0$, а другой — исход $1$. Сравните результат со средним ожидаемым регретом алгоритма взвешенного большинства, который не обладает информацией о том, что исходы независимы и одинаково распределены. Можно ли с течением времени для алгоритма взвешенного большинства сделать средний ожидаемый регрет сколь угодно малым? Какой алгоритм оказался лучше в итоге?

\subsection{Решение}

Предлагаемый алгоритм заключается в том, чтобы на каждом шаге выбирать исход, который наиболее вероятен, исходя из имеющейся истории. То есть на каждом шаге алгоритм выбирает исход $\hat{y_t}$, который минимизирует ожидаемый регрет $E[l(y_t,\hat{y_t})]$ по всем возможным исходам $\hat{y_t}$. Формально, для каждого $t\geq 1$:

$$\hat{y_t} = argmin_{y \in {0, 1}}\mathbb{E}\left[l(y, y_t|y_1, \ldots, y_{t-1})\right]$$

Ожидаемый регрет для данного алгоритма можно оценить следующим образом:
$$R(T) = \mathbb{E}\left[\sum_{t=1}^{T} l\left(y, \hat{y_t}\right)\right] \le \sum_{t=1}^{T}\mathbb{E}\left[l\left(y, \hat{y_t}\right)\right]$$

Далее оценим ожидаемый регрет по сравнению с двумя экспертами. Предположим, что вероятность исхода 1 равна $p$. Тогда для эксперта, всегда выбирающего исход 0, и эксперта, всегда выбирающего исход 1, ожидаемый регрет равен соответственно:

$$R_0(T) = \mathbb{E}\left[\sum_{t=1}^{T} l\left(y_t, 0\right)\right] = Tp$$

$$R_0(T) = \mathbb{E}\left[\sum_{t=1}^{T} l\left(y_t, 1\right)\right] = T(1 - p)$$

Тогда оценка среднего ожидаемого регрета для нашего алгоритма будет следующей:

$$R_0(T) \le \sum_{t=1}^{T}\mathbb{E}\left[l\left(y, \hat{y_t}\right)\right] = \sum_{t=1}^{T}\mathbb{E}\left|\hat{p_t} - p\right| = \sum_{t=1}^{T}\mathbb{E}\left|\hat{p_t} - \frac{1}{2}\right| + \frac{1}{2}$$

где $\hat{p_t}$ - это оценка вероятности исхода 1 на момент времени $t$.

\end{document}
