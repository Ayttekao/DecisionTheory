\documentclass[a4paper,14pt]{extarticle}
\usepackage[T2A]{fontenc}
\usepackage[utf8]{inputenc}
\usepackage[russian]{babel}

\usepackage[dvips]{graphicx}
\usepackage{color}
\usepackage[dvips]{hyperref}

\usepackage{setspace}
\usepackage{indentfirst}
\usepackage{textcomp}
\usepackage{ifthen}
\usepackage{calc}

\usepackage{amssymb}

\usepackage[cache=false]{minted}


\begin{document}
\newpage

\part{Теоретическая}
\section{Задание \textnumero 1}

Предположим, что вам необходимо предсказать последовательность $Y_1, Y_2, \ldots \in {0, 1}$ независимых и одинаково распределенных случайных величин с неизвестным распределением. Пусть $D = [0, 1]$, $l(x, y) = |x - y|$. Придумайте свой алгоритм для решения этой задачи. Оцените средний ожидаемый регрет своего алгоритма по сравнению с двумя экспертами, один из которых всегда выбирает исход $0$, а другой — исход $1$. Сравните результат со средним ожидаемым регретом алгоритма взвешенного большинства, который не обладает информацией о том, что исходы независимы и одинаково распределены. Можно ли с течением времени для алгоритма взвешенного большинства сделать средний ожидаемый регрет сколь угодно малым? Какой алгоритм оказался лучше в итоге?

\subsection{Решение}

Используя неравенство Хефдинга, мы можем оценить средний ожидаемый регрет для каждого из алгоритмов.
Для алгоритма, который всегда выбирает 0 или 1, средний ожидаемый регрет равен:

$$
R_{\exp }=E[\max (p, 1-p)]=\max (p, 1-p)
$$

где $p$ - вероятность правильного ответа.

Для алгоритма взвешенного большинства, который не знает, что исходы независимо и одинаково распределены, средний ожидаемый регрет можно оценить следующим образом:

$$
R_{\mathrm{maj}}=E\left|p-\hat{p}\right| \leq \sqrt{\frac{\ln (2 / \delta)}{2 n}}
$$

где $\hat{p}$ - доля правильных ответов, $n$ - количество предсказаний, а $\delta$ - параметр доверительного интервала.

Для алгоритма, который выбирает случайное значение с вероятностью 0.5, средний ожидаемый регрет также можно оценить с помощью неравенства Хефдинга:

$$
R_{\mathrm{rand}}=E[\max (p, 1-p, 0.5)] \leq \frac{1}{2}
$$

В итоге, наилучшим алгоритмом оказался алгоритм взвешенного большинства, который может достичь произвольно маленького среднего ожидаемого регрета при настройке весов и определении вероятностей правильных ответов.
\end{document}
