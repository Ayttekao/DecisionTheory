\documentclass[a4paper,14pt]{extarticle}
\usepackage[T2A]{fontenc}
\usepackage[utf8]{inputenc}
\usepackage[russian]{babel}

\usepackage[dvips]{graphicx}
\usepackage{color}
\usepackage[dvips]{hyperref}

\usepackage{setspace}
\usepackage{indentfirst}
\usepackage{textcomp}
\usepackage{ifthen}
\usepackage{calc}

\usepackage{amssymb}

\usepackage[cache=false]{minted}


\begin{document}
\newpage

\part{Теоретическая}
\section{Задание \textnumero 1}

Предположим, что вам необходимо предсказать последовательность $Y_1, Y_2, \ldots \in {0, 1}$ независимых и одинаково распределенных случайных величин с неизвестным распределением. Пусть $D = [0, 1]$, $l(x, y) = |x - y|$. Придумайте свой алгоритм для решения этой задачи. Оцените средний ожидаемый регрет своего алгоритма по сравнению с двумя экспертами, один из которых всегда выбирает исход $0$, а другой — исход $1$. Сравните результат со средним ожидаемым регретом алгоритма взвешенного большинства, который не обладает информацией о том, что исходы независимы и одинаково распределены. Можно ли с течением времени для алгоритма взвешенного большинства сделать средний ожидаемый регрет сколь угодно малым? Какой алгоритм оказался лучше в итоге?

\subsection{Решение}

Для решения задачи мы можем использовать следующий алгоритм:
\begin{itemize}
\item Сгенерировать случайное число $R$ из диапазона $[0,1]$.
\item Если $R$ меньше $0.5$, выбрать исход 0, иначе выбрать исход 1.
\end{itemize}

Такой алгоритм не зависит от распределения случайных величин $Y_1$, $Y_2$, ... и является случайным предсказанием, которое не учитывает никаких особенностей данных.

Оценим средний ожидаемый регрет данного алгоритма по сравнению с двумя экспертами, один из которых всегда выбирает исход 0, а другой — исход 1. Пусть $p$ — вероятность того, что исход 1 является правильным ответом. Тогда средний ожидаемый регрет данного алгоритма равен:
$$ E\left[R\right] = p \cdot \left|0 - \frac{1}{2}\right| + \left(1 - p\right) \cdot \left|1 - \frac{1}{2}\right| = \left|\frac{1}{2} - p\right| $$

В то же время, если мы всегда выбираем эксперта, который правильно угадывает исход, то средний ожидаемый регрет равен:

$$ E[R] = min\left(p, 1 - p\right)$$

Теперь рассмотрим алгоритм взвешенного большинства. Пусть $w_i$ — вес, который мы присваиваем эксперту $i$ ($i = 0, 1$). Тогда мы можем предсказать исход с наибольшей вероятностью как:

$$
\arg \max _i\left\{w_i\right\}
$$

Если мы не знаем распределение случайных величин, то мы можем задать веса равными 1 и получить алгоритм равных голосов. Тогда средний ожидаемый регрет алгоритма взвешенного большинства равен:

$$
E[R]=\min (p, 1-p)+1 / 2-\max (p, 1-p)
$$

Как видим, этот алгоритм обладает меньшим регретом, чем случайное предсказание.

Отметим, что взвешенное большинство может иметь произвольно малый регрет с течением времени, если мы сможем точно определить вероятности $p$ и присвоить веса экспертам в соответствии с этими вероятностями. В этом случае алгоритм взвешенного большинства становится оптимальным.

Таким образом, мы можем сделать вывод, что алгоритм взвешенного большинства обладает меньшим регретом, чем случайное предсказание, и что с течением времени его регрет может быть сколь угодно малым при определении вероятностей и правильной настройке весов экспертов. Поэтому, в данном случае, алгоритм взвешенного большинства является лучшим выбором.

Используя неравенство Хефдинга, мы можем оценить средний ожидаемый регрет для каждого из алгоритмов.
Для алгоритма, который всегда выбирает 0 или 1, средний ожидаемый регрет равен:

$$
R_{\exp }=E[\max (p, 1-p)]=\max (p, 1-p)
$$

где $p$ - вероятность правильного ответа.

Для алгоритма взвешенного большинства, который не знает, что исходы независимо и одинаково распределены, средний ожидаемый регрет можно оценить следующим образом:

$$
R_{\mathrm{maj}}=E\left|p-\hat{p}\right| \leq \sqrt{\frac{\ln (2 / \delta)}{2 n}}
$$

где $\hat{p}$ - доля правильных ответов, $n$ - количество предсказаний, а $\delta$ - параметр доверительного интервала.

Для алгоритма, который выбирает случайное значение с вероятностью 0.5, средний ожидаемый регрет также можно оценить с помощью неравенства Хефдинга:

$$
R_{\mathrm{rand}}=E[\max (p, 1-p, 0.5)] \leq \frac{1}{2}
$$

В итоге, наилучшим алгоритмом оказался алгоритм взвешенного большинства, который может достичь произвольно маленького среднего ожидаемого регрета при настройке весов и определении вероятностей правильных ответов.

\end{document}
